\documentclass[xcolor={svgnames,rgb},aspectratio=1610]{ctexbeamer}
% xcolor: e.g. provides \colorlet
% dvipsnames option: extra named colours
% aspectratio=1610: for modern widescreen monitors...
% see https://www.overleaf.com/learn/latex/Using_colours_in_LaTeX for more detail

% equation annotations also work in articles, of course (then might need explicit \usepackage[dvipsnames]{xcolor}

%%% standard math packages for equations:

\usepackage{amsmath}
\usepackage{amssymb}
\usepackage{mathtools}

\usepackage{annotate-equations}

\begin{document}
    % \colorlet{Mycolor1}{green!10!orange}: A color named Mycolor1 is created with 10% green and 90% orange.
    % =============== dvipsnames
    % \colorlet{colorp}{NavyBlue}
    % \colorlet{colorT}{WildStrawberry}
    % \colorlet{colork}{ForestGreen}%OliveGreen
    % \colorlet{colorM}{RoyalPurple}
    % \colorlet{colorNb}{Plum}
    % \colorlet{colorIs}{black}
    % =============== dvipsnames

    % =============== svgnames
    \colorlet{color1}{MediumBlue} % DodgerBlue DeepSkyBlue
    \colorlet{color2}{BlueViolet}
    \colorlet{color3}{Green}%OliveGreen
    \colorlet{color4}{LightPink}
    \colorlet{color5}{Aqua}
    \colorlet{color6}{MediumTurquoise}
\begin{frame}
    \LARGE %%% increase font size to make equation more readable!

    %%% we define our own colors upfront - this makes it easier to keep it consistent if you change your mind


     % Plum
     % NavyBlue
     % Bittersweet
     % xkcdHunterGreen


    \begin{equation*}
        \color{Black}
        \mathcal{O}\big(
            (
            % \tikzmarknode is what links parts of the equation and corresponding annotations
            \eqnmarkbox[color1]{p1}{p}
            \eqnmarkbox[color2]{k1}{\kappa}^3 % note that we have the ^3 outside the \tikzmarknode
            )
            \eqnmarkbox{T1}{T}
            +
            (
            \eqnmarkbox[color3]{p2}{p} % tikzmarks need distinct names!
            \eqnmark[color4]{k2}{\kappa}
            )
            (
            \eqnmarkbox[color5]{T2}{T}^2
            \tikzmarknode{Is}{|\mathcal{I}^*|}
            \eqnmarkbox[color6]{Nb}{N}_{\!\!\eqnmarkbox[DarkOrange]{b}{b}}
            \eqnmark[color1]{M}{M}
            )
        \big)
    \end{equation*}
    \annotatetwo[yshift=1em]{above}{p1}{p2}{\# of nodes}
    \annotatetwo[yshift=-1em,xshift=0.2ex]{below}{T1}{T2}{\# of graphs in $\hat{\mathcal{G}}_T$}
    \annotatetwo[yshift=-2em]{below}{k1}{k2}{max.\ indegree in $\hat{\mathcal{G}}_T$}
    \annotate[yshift=3em]{above,left}{Is}{size of set of allowed interventions}
    \annotate[yshift=1em]{above}{Nb}{\# of samples per batch}
    \annotate[yshift=-1em]{below}{M}{\# of samples for $\mathbb{E}_y$}
    \annotate[yshift=-3em]{below,left}{b}{batch index}
\end{frame}

\begin{frame}{测试}

    \begin{equation*}
    \eqnmarkbox[color1]{O}{\boldsymbol{O}} =
    \eqnmarkbox[color2]{I}{\boldsymbol{I}}
    \eqnmarkbox[color3]{Conv}{\circledast} 
    \eqnmarkbox[color4]{F}{\boldsymbol{F}} +
    \eqnmarkbox[color6]{REP}{\operatorname{REP}(\boldsymbol{b})}
    \end{equation*}

    \annotate[yshift=4em]{above,right}{O}{output featuremap of D-channdel}
    \annotate[yshift=2em]{above,right}{I}{input featuremap of C-channdel}
    \annotate[yshift=-1em]{below,left}{Conv}{convolution}
    \annotate[yshift=-3em]{below,left}{F}{parameters of conv}
    \annotate[yshift=-5em]{below,left}{REP}{XXXXXXXX}
    
\end{frame}
\end{document}
